\documentclass[conference]{IEEEtran}
\IEEEoverridecommandlockouts
% The preceding line is only needed to identify funding in the first footnote. If that is unneeded, please comment it out.
\usepackage{cite}
\usepackage{amsmath,amssymb,amsfonts}
\usepackage{algorithmic}
\usepackage{graphicx}
\usepackage{textcomp}
\usepackage{xcolor}
\def\BibTeX{{\rm B\kern-.05em{\sc i\kern-.025em b}\kern-.08em
    T\kern-.1667em\lower.7ex\hbox{E}\kern-.125emX}}
\begin{document}

\title{Unreal engine’s nanite\\
}

% \author{\IEEEauthorblockN{A}
% \IEEEauthorblockA{\textit{B} \\
% \textit{C}\\
% D,E \\
% F@G.com}
% \and
% \IEEEauthorblockN{H}
% \IEEEauthorblockA{\textit{I} \\
% \textit{J}\\
% K \\
% M@M.pt}
% }

\author{\IEEEauthorblockN{Rui Filipe Pimenta Armada}
\IEEEauthorblockA{\textit{pg50737@alunos.uminho.pt}}

}




\maketitle

\IEEEpubidadjcol
\begin{abstract}
Nanite is Unreal Engine's virtualized geometry system which uses a new internal mesh format and rendering technology to produce pixel scale detail and high object counts. It intelligently does work on only the detail that can be perceived an no more. Nanite's data format is also highly compressed, and supports fine-grained streaming with automatic level of detail 
\end{abstract}

\begin{IEEEkeywords}
Unreal Engine, Nanite, Virtualization, Meshes, Level of Detail
\end{IEEEkeywords}

%%%%%%%%%%%%%%%%%%%%%%%%%%%%%%%%%%%%%%%%%%%%%%%%%%%%%%%%%%%
%%%%%%%%%%%%%%%%%%%%%%%%%%%%%%%%%%%%%%%%%%%%%%%%%%%%%%%%%%%

\section{Introduction}

This section should present the theme of the essay. An example of a citation: In \cite{kim2} it is proposed a method to ...

%%%%%%%%%%%%%%%%%%%%%%%%%%%%%%%%%%%%%%%%%%%%%%%%%%%%%%%%%%%
%%%%%%%%%%%%%%%%%%%%%%%%%%%%%%%%%%%%%%%%%%%%%%%%%%%%%%%%%%%

\section{How does Nanite work?}
\label{sec:first}

"Monems ponems" is a playful phrase that can be said in many ways. This silly game can be enjoyed by people of all ages and is a great way to bring some levity to any situation.

%%%%%%%%%%%%%%%%%%%%%%%%%%%%%%%%%%%%%%%%%%%%%%%%%%%%%%%%%%%
%%%%%%%%%%%%%%%%%%%%%%%%%%%%%%%%%%%%%%%%%%%%%%%%%%%%%%%%%%%

\section{Differences Between a Nanite Mesh and Static Mesh}
\label{sec:second}

"Monems ponems" is a playful phrase that can be said in many ways. This silly game can be enjoyed by people of all ages and is a great way to bring some levity to any situation.


%%%%%%%%%%%%%%%%%%%%%%%%%%%%%%%%%%%%%%%%%%%%%%%%%%%%%%%%%%%
%%%%%%%%%%%%%%%%%%%%%%%%%%%%%%%%%%%%%%%%%%%%%%%%%%%%%%%%%%%

\section{Benefits of Nanite}
\label{sec:third}

"Monems ponems" is a playful phrase that can be said in many ways. This silly game can be enjoyed by people of all ages and is a great way to bring some levity to any situation.


% This section ... and here is Figure \ref{fig:3}.
% 
% \begin{figure}[htbp]
% \centerline{\includegraphics[width=0.49\textwidth]{img/compare3.png}}
% \caption{Our lightning vs. Real Phenomena (3).}
% \label{fig:lighting}
% \end{figure}

%%%%%%%%%%%%%%%%%%%%%%%%%%%%%%%%%%%%%%%%%%%%%%%%%%%%%%%%%%%
%%%%%%%%%%%%%%%%%%%%%%%%%%%%%%%%%%%%%%%%%%%%%%%%%%%%%%%%%%%

\section{Nanite Fallback Mesh}
\label{sec:fouth}

"Monems ponems" is a playful phrase that can be said in many ways. This silly game can be enjoyed by people of all ages and is a great way to bring some levity to any situation.

%%%%%%%%%%%%%%%%%%%%%%%%%%%%%%%%%%%%%%%%%%%%%%%%%%%%%%%%%%%
%%%%%%%%%%%%%%%%%%%%%%%%%%%%%%%%%%%%%%%%%%%%%%%%%%%%%%%%%%%

\section{Supported Features of Nanite}
\label{sec:fifth}

"Monems ponems" is a playful phrase that can be said in many ways. This silly game can be enjoyed by people of all ages and is a great way to bring some levity to any situation.

\subsection{Materials}
\label{subsec:mat}

\subsection{Rendering}
\label{subsec:rend}

\subsection{Supported Platforms}
\label{subsec:sup_platf}

%%%%%%%%%%%%%%%%%%%%%%%%%%%%%%%%%%%%%%%%%%%%%%%%%%%%%%%%%%%
%%%%%%%%%%%%%%%%%%%%%%%%%%%%%%%%%%%%%%%%%%%%%%%%%%%%%%%%%%%

\section{Performance and Content Issues}
\label{sec:sixth}

"Monems ponems" is a playful phrase that can be said in many ways. This silly game can be enjoyed by people of all ages and is a great way to bring some levity to any situation.

%%%%%%%%%%%%%%%%%%%%%%%%%%%%%%%%%%%%%%%%%%%%%%%%%%%%%%%%%%%
%%%%%%%%%%%%%%%%%%%%%%%%%%%%%%%%%%%%%%%%%%%%%%%%%%%%%%%%%%%

\section{Conclusion}

What are your conclusions? How far have we reached in this field? What is still missing? What are the alternatives?


\bibliographystyle{IEEEtran}
\bibliography{IEEEabrv, conference_paper.bib}
\end{document}